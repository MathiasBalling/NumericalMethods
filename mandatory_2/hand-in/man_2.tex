\documentclass{article}

\usepackage[a4paper, total={6in, 9in}]{geometry}

\usepackage{amsmath} % For math
\usepackage{amssymb} % For math
\usepackage{amsthm} % For math
\usepackage{esint} % For math
\usepackage{siunitx} % For SI units

\usepackage{graphicx} % For figures
\usepackage{subcaption} % For figure

\usepackage{float} % For figure placement

\usepackage{minted} % For code

\usepackage{hyperref} % For links
\hypersetup{
	colorlinks,
	citecolor=black,
	filecolor=black,
	linkcolor=black,
	urlcolor=blue,
}
\usepackage{cleveref} % For references

\usepackage{parskip} % For paragraph spacing and no indentation

% \usepackage{tabularx} % For tables
\usepackage[table,xcdraw]{xcolor} % For tables
\def\arraystretch{1.5} % Better looking tables
\usepackage{adjustbox}

% Costum commands:

\newcommand{\qr}{{\quad\Rightarrow\quad}}
\newcommand{\qqr}{{\qquad\Rightarrow\qquad}}



\title{Mandatory 2}
\author{Mathias Balling}
\date{\today}

\begin{document}
\maketitle
% \tableofcontents
% \newpage

\section{Exercise 1}
NR3's implementation of \texttt{newt} from "roots\_multidim.h" is used as a base for solving the non-linear system of equations.

The vector function is set up as follows:
\begin{minted}{cpp}
// Material constants:
const double v = 120.0;                 // kg
const double k = 2.5;                   // m
const double w = 4.0;                   // kg/m
const double alpha = 2.0 * pow(10, -7); // kg^-1

// Other constants:
const double d = 30.0; // m
double n = 5.0;        // m (Updated in main)

VecDoub vecfunc(VecDoub_I q) {
  assert(q.size() == 8);

  // q=[L0, L, p, x, theta, phi/varphi, a, H]
  // Zero vectors
  VecDoub f(8);
  f[0] = (q[6] * (cosh(q[3] / q[6]) - 1.0)) - q[2];
  f[1] = (2.0 * q[6] * sinh(q[3] / q[6])) - q[1];
  f[2] = (2.0 * q[3] + 2.0 * k * cos(q[4])) - d;
  f[3] = (q[2] + k * sin(q[4])) - n;
  f[4] = (sinh(q[3] / q[6])) - tan(q[5]);
  f[5] = ((1.0 + (v / (w * q[0]))) * tan(q[5])) - tan(q[4]);
  f[6] = (q[0] * (1.0 + alpha * q[7])) - q[1];
  f[7] = ((w * q[0]) / (2.0 * sin(q[5]))) - q[7];

  return f;
}
\end{minted}

\subsection{}
For $n=5$:
$$
\begin{array}{ccc}
  L_0 & = & 27.5233 \\
  H & = & 124.541
\end{array}
$$
using starting guess:
$$
\begin{array}{cccccccc}
L_0 = 29 & L = 29.1 & p = 5 & x = 15 & \theta  = 1 & \varphi = 0.5 & a = 40 & H = 100
\end{array}
$$

\subsection{}
For $n=2$:
$$
\begin{array}{ccc}
  L_0 & = & 25.4565 \\
  H & = & 294.733
\end{array}
$$
using starting guess:
$$
\begin{array}{cccccccc}
L_0 = 27 & L = 27.1 & p = 2 & x = 14 & \theta  = 1 & \varphi = 0.5 & a = 40 & H = 100
\end{array}
$$

\subsection{}
For $n=1$:
$$
\begin{array}{ccc}
  L_0 & = & 25.114 \\
  H & = & 587.781
\end{array}
$$
using starting guess:
$$
\begin{array}{cccccccc}
L_0 = 26 & L = 26.1 & p = 1 & x = 13 & \theta  = 0.5 & \varphi = 0.5 & a = 40 & H = 100
\end{array}
$$

\subsection{}
For $n=0.5$:
$$
\begin{array}{ccc}
  L_0 & = & 25.0235 \\
  H & = & 1174.94
\end{array}
$$
using starting guess:
$$
\begin{array}{cccccccc}
L_0 = 26 & L = 26.1 & p = 0.5 & x = 13 & \theta  = 0.2 & \varphi = 0.1 & a = 40 & H = 100
\end{array}
$$

\subsection{}
For $n=0.2$:
$$
\begin{array}{ccc}
  L_0 & = & 24.99 \\
  H & = & 2936.27
\end{array}
$$
using starting guess:
$$
\begin{array}{cccccccc}
L_0 = 25.5 & L = 25.6 & p = 0.2 & x = 12 & \theta  = 0.1 & \varphi = 0.05 & a = 40 & H = 100
\end{array}
$$

\subsection{}
For $n=0.1$:
$$
\begin{array}{ccc}
  L_0 & = & 24.9719\\
  H & = & 5869.9
\end{array}
$$
using starting guess:
$$
\begin{array}{cccccccc}
L_0 = 25.5 & L = 25.6 & p = 0.1 & x = 12 & \theta  = 0.1 & \varphi = 0.05 & a = 40 & H = 100
\end{array}
$$
\end{document}

\documentclass{article}

\usepackage[a4paper, total={6in, 9in}]{geometry}

\usepackage{amsmath} % For math
\usepackage{amssymb} % For math
\usepackage{amsthm} % For math
\usepackage{esint} % For math
\usepackage{siunitx} % For SI units

\usepackage{graphicx} % For figures
\usepackage{subcaption} % For figure

\usepackage{float} % For figure placement

\usepackage{minted} % For code

\usepackage{hyperref} % For links
\hypersetup{
	colorlinks,
	citecolor=black,
	filecolor=black,
	linkcolor=black,
	urlcolor=blue,
}
\usepackage{cleveref} % For references

\usepackage{parskip} % For paragraph spacing and no indentation

% \usepackage{tabularx} % For tables
\usepackage[table,xcdraw]{xcolor} % For tables
\def\arraystretch{1.5} % Better looking tables
\usepackage{adjustbox}

% Costum commands:

\newcommand{\qr}{{\quad\Rightarrow\quad}}
\newcommand{\qqr}{{\qquad\Rightarrow\qquad}}



\title{Mandatory 3}
\author{Mathias Balling}
\date{\today}

\begin{document}
\maketitle
% \tableofcontents
% \newpage

\section*{Exercise 5}
NR3's implementation of \texttt{DErule} from "derule.h" is used as a base for solving the integral in question 5.3

The equations for Extended Midpoint and DErule are set up as follows:
\begin{minted}{cpp}
// For extended midpoint
double eqn(double x) { return (cos(pow(x, 3)) * exp(-x)) / sqrt(x); }

// For DErule
double eqn_derule(double x, double delta) {
  // If x is small use delta instead to avoid division by zero
  if (abs(x) < 1e-6) {
    return (cos(pow(x, 3)) * exp(-x)) / sqrt(delta);
  } else {
    return (cos(pow(x, 3)) * exp(-x)) / sqrt(x);
  }
}
\end{minted}

\subsection*{5.1}
The analytical solution using Extended Midpoint is:
$$
\int_{a}^{b}\frac{\cos(x^3)\exp(-x)}{\sqrt{x}}dx
$$
$$
\approx
  h\cdot \sum_{i=0}^{N-2}\left(
  \frac{
  \cos\left(\left(a+h\cdot i+\frac{h}{2}\right)^3\right)\exp\left(-(a+h\cdot i+\frac{h}{2})\right)
}{
  \sqrt{(a+h\cdot i+\frac{h}{2})}
}
\right)+O\left(\frac{1}{N^2}\right)
\quad\text{where } \begin{array}{l}
  h=\frac{(b-a)}{N-1}\\
\end{array}
$$

$$
=(b-a)\cdot\left(
  \frac{
  \cos\left(\left(a+\frac{(b-a)}{2-1}\cdot 0+\frac{\frac{(b-a)}{2-1}}{2}\right)^3\right)\exp\left(-(a+\frac{(b-a)}{2-1}\cdot 0+\frac{\frac{(b-a)}{2-1}}{2})\right)
}{
  \sqrt{a+\frac{(b-a)}{2-1}\cdot 0+\frac{\frac{(b-a)}{2-1}}{2}}
}
\right)+O\left(\frac{1}{2^2}\right)
\quad{\text{for } N=2}
$$
$$
=(b-a)\cdot\left(
  \frac{
  \cos\left(\left(a+\frac{b-a}{2}\right)^3\right)\exp\left(-a-\frac{b-a}{2}\right)
}{
  \sqrt{a+\frac{b-a}{2}}
}
\right)+O\left(\frac{1}{4}\right)
$$

\subsection*{5.2}
The accuracy is computed as follows using Richardson Extrapolation Error:
\begin{minted}{cpp}
std::vector<double> 
richardson_extrapolation_error(const std::vector<double> &A_k,
                               const double alpha_k_order_expected) {
  std::vector<double> A_R = {NAN}; // No error on the first
  for (size_t i = 1; i < A_k.size(); i++) {
    const double A_1 = A_k[i - 1];
    const double A_2 = A_k[i];
    // pow(2, alpha_k_order) as we use N-1=1,2,4,8
    double error = (A_2 - A_1) / (pow(2, alpha_k_order_expected) - 1);
    A_R.push_back(error);
  }
  return A_R;
}
\end{minted}

The generated table is (using 'utils/quadrature\_table.h'):
\begin{minted}{bash}
////////////////////////////////////// Extended Midpoint: //////////////////////////////
|  i   |      A(i)    | A(i-1)-A(i)  |  alpha^k   | Rich error | Order est. | f comps  |
|------|--------------|--------------|------------|------------|------------|----------|
|  1   |  -0.055695   |              |            |            |            |    1     |
|  2   |   0.380737   |   -0.436432  |            |  0.145477  |            |    2     |
|  3   |   0.629306   |   -0.248568  |  1.755781  |  0.082856  |   0.81211  |    4     |
|  4   |   1.010516   |   -0.381209  |  0.652052  |  0.127069  |  -0.61693  |    8     |
|  5   |   1.135519   |   -0.125003  |  3.049604  |  0.041667  |   1.60862  |    16    |
|  6   |   1.214732   |  -0.0792131  |  1.578058  |  0.026404  |   0.65815  |    32    |
|  7   |   1.277964   |  -0.0632325  |  1.252728  |  0.021077  |   0.32507  |    64    |
|  8   |   1.322794   |  -0.0448294  |  1.410512  |  0.014943  |   0.49621  |   128    |
|  9   |   1.354316   |  -0.0315223  |  1.422148  |  0.010507  |   0.50807  |   256    |
|  10  |   1.376536   |  -0.0222198  |  1.418658  |  0.007406  |   0.50452  |   512    |
|  11  |   1.392222   |  -0.0156861  |  1.416521  |  0.005228  |   0.50235  |   1024   |
|  12  |   1.403305   |  -0.0110825  |  1.415391  |  0.003694  |   0.50120  |   2048   |
|  13  |   1.411138   |  -0.0078332  |  1.414810  |  0.002611  |   0.50060  |   4096   |
|  14  |   1.416676   |  -0.0055377  |  1.414514  |  0.001845  |   0.50030  |   8192   |
|  15  |   1.420591   |  -0.0039153  |  1.414364  |  0.001305  |   0.50015  |  16384   |
|  16  |   1.423360   |  -0.0027684  |  1.414289  |  0.000922  |   0.50007  |  32768   |
\end{minted}

The function computations for each N-1=pow(2,i) is shown in the table.

\subsection*{5.3}
DErule found the integral to be:
$$
1.4300433455
$$
using 127 f-calculation.

\end{document}
